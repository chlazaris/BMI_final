%%%%%%%%%%%%%%%%%%%%%%%%%%%%%%%%%%%%%%%%%
% Thin Sectioned Essay
% LaTeX Template
% Version 1.0 (3/8/13)
%
% This template has been downloaded from:
% http://www.LaTeXTemplates.com
%
% Original Author:
% Nicolas Diaz (nsdiaz@uc.cl) with extensive modifications by:
% Vel (vel@latextemplates.com)
%
% License:
% CC BY-NC-SA 3.0 (http://creativecommons.org/licenses/by-nc-sa/3.0/)
%
%%%%%%%%%%%%%%%%%%%%%%%%%%%%%%%%%%%%%%%%%

%----------------------------------------------------------------------------------------
%	PACKAGES AND OTHER DOCUMENT CONFIGURATIONS
%----------------------------------------------------------------------------------------

\documentclass[a4paper, 11pt]{article} % Font size (can be 10pt, 11pt or 12pt) and paper size (remove a4paper for US letter paper)

\usepackage[protrusion=true,expansion=true]{microtype} % Better typography
\usepackage{graphicx} % Required for including pictures
\usepackage{wrapfig} % Allows in-line images

\usepackage{mathpazo} % Use the Palatino font
\usepackage[T1]{fontenc} % Required for accented characters
\linespread{1.05} % Change line spacing here, Palatino benefits from a slight increase by default

\makeatletter
\renewcommand\@biblabel[1]{\textbf{#1.}} % Change the square brackets for each bibliography item from '[1]' to '1.'
\renewcommand{\@listI}{\itemsep=0pt} % Reduce the space between items in the itemize and enumerate environments and the bibliography

\renewcommand{\maketitle}{ % Customize the title - do not edit title and author name here, see the TITLE block below
\begin{flushright} % Right align
{\LARGE\@title} % Increase the font size of the title

\vspace{50pt} % Some vertical space between the title and author name

{\large\@author} % Author name
\\\@date % Date

\vspace{40pt} % Some vertical space between the author block and abstract
\end{flushright}
}

%----------------------------------------------------------------------------------------
%	TITLE
%----------------------------------------------------------------------------------------

\title{\textbf{Studying three-dimensional genome organization using Hi-C}}\\ % Title
%Evaluation of state-of-the-art methods} % Subtitle

\author{\textsc{Harris A. Lazaris}\\ % Author
{\textit{BMI Final Project}}} % Course

\date{\today} % Date

%----------------------------------------------------------------------------------------

\begin{document}

\maketitle % Print the title section

%----------------------------------------------------------------------------------------
%	ABSTRACT AND KEYWORDS
%----------------------------------------------------------------------------------------

\renewcommand{\abstractname}{Summary} % Uncomment to change the name of the abstract to something else

\begin{abstract}
Morbi tempor congue porta. Proin semper, leo vitae faucibus dictum, metus mauris lacinia lorem, ac congue leo felis eu turpis. Sed nec nunc pellentesque, gravida eros at, porttitor ipsum. Praesent consequat urna a lacus lobortis ultrices eget ac metus. In tempus hendrerit rhoncus. Mauris dignissim turpis id sollicitudin lacinia. Praesent libero tellus, fringilla nec ullamcorper at, ultrices id nulla. Phasellus placerat a tellus a malesuada.
\end{abstract}

\hspace*{3,6mm}\textit{Keywords:} Hi--C , 3D genome organization , evaluation  % Keywords

\vspace{30pt} % Some vertical space between the abstract and first section

%----------------------------------------------------------------------------------------
%	ESSAY BODY
%----------------------------------------------------------------------------------------

\section*{Introduction}

\subsection*{Background information}

Background information goes here~\ldots

\subsection*{Motivation}

Motivation goes here~\ldots

This statement requires citation \cite{Smith:2012qr}; this one does too \cite{Smith:2013jd}. Lorem ipsum dolor sit amet, consectetur adipiscing elit. Aenean dictum lacus sem, ut varius ante dignissim ac. Sed a mi quis lectus feugiat aliquam. Nunc sed vulputate velit. Sed commodo metus vel felis semper, quis rutrum odio vulputate. Donec a elit porttitor, facilisis nisl sit amet, dignissim arcu. Vivamus accumsan pellentesque nulla at euismod. Duis porta rutrum sem, eu facilisis mi varius sed. Suspendisse potenti. Mauris rhoncus neque nisi, ut laoreet augue pretium luctus. Vestibulum sit amet luctus sem, luctus ultrices leo. Aenean vitae sem leo.

Nullam semper quam at ante convallis posuere. Ut faucibus tellus ac massa luctus consectetur. Nulla pellentesque tortor et aliquam vehicula. Maecenas imperdiet euismod enim ut pharetra. Suspendisse pulvinar sapien vitae placerat pellentesque. Nulla facilisi. Aenean vitae nunc venenatis, vehicula neque in, congue ligula.

Pellentesque quis neque fringilla, varius ligula quis, malesuada dolor. Aenean malesuada urna porta, condimentum nisl sed, scelerisque nisi. Suspendisse ac orci quis massa porta dignissim. Morbi sollicitudin, felis eget tristique laoreet, ante lacus pretium lacus, nec ornare sem lorem a velit. Pellentesque eu erat congue, ullamcorper ante ut, tristique turpis. Nam sodales mi sed nisl tincidunt vestibulum. Interdum et malesuada fames ac ante ipsum primis in faucibus.

%------------------------------------------------

\section*{Materials-Methods}



%------------------------------------------------


\section*{Results}

\subsection*{Complexity--Timing Analysis}

\subsection*{Focus area}

I focused on the use of Python packages (Numpy, Pandas)
as well as on visualization (with matplotlib and Pandas). 


% \begin{wrapfigure}{l}{0.4\textwidth} % Inline image example
% \begin{center}
% \includegraphics[width=0.38\textwidth]{fish.png}
% \end{center}
% \caption{Fish}
% \end{wrapfigure}


%------------------------------------------------

\section*{Conclusion-Discussion}

It is clear from the analysis presented in this study that
even state-of-the-art Hi--C analysis techniques do not
give extremely reproducible results. When two different
restriction enzymes are used with the same sample as source,
while the correlation is relatively high, it is not ideal.
Moreover, when the resolution is increased (128kb bins instead
of 4096kb bins), the correlation even in the case of technical
replicates treated with the same enzymes is very low. Thus, new
more robust analysis techniques that lead to more reproducible
results, are required. This is going to be a large part of my 
PhD thesis work.

While I optimized the code for speed, by using Numpy instead of
native Python to deal with matrices and list comprehension insted
of for loops when possible, the real bottleneck is the I/O operations
(reading input and writing output). This is critical as the matrix
files that I have to deal with, are really large. One solution to
the problem would be to minimize I/O operations by using Numpy
for everything, but this would require much better knowledge
of Numpy than I currently have. Moreover, Numpy may be not as versatile
or efficient as R in certain circumstances which means that 
I may be unable to fully replace R with Numpy. Another approach 
would be to use rpy or rpy2 for R/Python integration but I have been
always facing problems with rpy/rpy2 installation on my system.

As far as the visualization is concerned, both Pandas and matplotlib
look interesting but:

\begin{enumerate}
\item They need time to learn.  
\item There are many requirements (dependencies etc) and in many
cases it is difficult to run code using these packages on the cluster.
\item The resulting graphs, especially in the case of Pandas, do not
seem to be extremely customizable. Moreover, I had to write more
code to achieve the same result I would get with writing less
code in R.
\end{enumerate}

I will certainly try to explore Numpy, matplotlib, Pandas and other
packages further, as it is always possible that the drawbacks I see
right now compared to R may be just due to luck of expertise on all
these Python packages. 

%-------------------------------------------------------------------------------------
%	BIBLIOGRAPHY
%-------------------------------------------------------------------------------------

\bibliographystyle{unsrt}

\bibliography{sample}

%-------------------------------------------------------------------------------------

\end{document}